
% This LaTeX was auto-generated from MATLAB code.
% To make changes, update the MATLAB code and republish this document.

\documentclass{article}
\usepackage{graphicx}
\usepackage{color}

\sloppy
\definecolor{lightgray}{gray}{0.5}
\setlength{\parindent}{0pt}

\begin{document}

    
    
\subsection*{Contents}

\begin{itemize}
\setlength{\itemsep}{-1ex}
   \item Alfabeto primario e testo primario
   \item Codifica con Huffman
   \item Codifica con Tunstall
   \item Confronto
\end{itemize}
\begin{verbatim}
% Questo script codifica un testo generato ad hoc in modo che la codifica
% di Tunstall risulti efficiente
%
\end{verbatim}


\subsection*{Alfabeto primario e testo primario}

\begin{verbatim}
c = 'AAAAAAAAAAAAAABACEAFGHIAAALMANAOPQARSTAAAUVAZAAAAAABAGAFFFARAVAAARAAFARATAY';

% Chiamo la funzione _findAlphabet_ che mi consente di trovare l'alfabeto
% dei simboli che occorrono nel testo primario e le relative frequenze
[A,P] = findAlphabet(c);

dimension = ceil(log2(length(A)))*length(c);

[Psort,Isort] = sort(P,'descend');
A(Isort);

% Converto il testo e l'alfabeto primari nel formato _cell_
symbols = zeros(length(A),1);
for i = 1:length(A)
    symbols(i) = A(i);
end
 symbols = cellstr(num2str(symbols));

 text = zeros(length(c),1);
 for i = 1:length(c)
     text(i) = c(i);
 end
 text = cellstr(num2str(text));
\end{verbatim}


\subsection*{Codifica con Huffman}

\begin{verbatim}
 tic
 dictH = huffmandict(symbols,P);
 toc

 tic
 compH = huffmanenco(text,dictH);
 toc
\end{verbatim}

        \color{lightgray} \begin{verbatim}Elapsed time is 0.033669 seconds.
Elapsed time is 0.015151 seconds.
\end{verbatim} \color{black}
    

\subsection*{Codifica con Tunstall}

\begin{verbatim}
 tic
 dictT = tunstall(symbols,P,1);
 toc

 tic
 compT = tunstallenco(text,dictT);
 toc
\end{verbatim}

        \color{lightgray} \begin{verbatim}Elapsed time is 0.010940 seconds.
Elapsed time is 0.159715 seconds.
\end{verbatim} \color{black}
    

\subsection*{Confronto}

\begin{verbatim}
 dimensionH = length(compH);
 rapportoHuffman = dimensionH/dimension
 dimensionT = length(compT);
 rapportoTunstall = dimensionT/dimension
\end{verbatim}

        \color{lightgray} \begin{verbatim}
rapportoHuffman =

    0.5387


rapportoTunstall =

    0.7200

\end{verbatim} \color{black}
    


\end{document}
    
