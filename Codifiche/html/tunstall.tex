
% This LaTeX was auto-generated from MATLAB code.
% To make changes, update the MATLAB code and republish this document.

\documentclass{article}
\usepackage{graphicx}
\usepackage{color}

\sloppy
\definecolor{lightgray}{gray}{0.5}
\setlength{\parindent}{0pt}

\begin{document}

    
    
\subsection*{Contents}

\begin{itemize}
\setlength{\itemsep}{-1ex}
   \item Creazione famiglia di messaggi
   \item Conversione in binario
\end{itemize}
\begin{verbatim}
function [T,PNew] = tunstall(A,P,j)
\end{verbatim}


\subsection*{Creazione famiglia di messaggi}

\begin{par}
Dato un alfabeto A,un vettore di probabilit� P e la lunghezza j, calcola la corrispondente codifica di Tunstall
\end{par} \vspace{1em}
\begin{verbatim}
M = A;
PNew = P;
for i = 1:j
    [~,maxIndex] = max(PNew);
    m = M{maxIndex}; %Seleziono l'elemento con probabilit� max
    M(maxIndex) = []; %Lo cancello
    p = PNew(maxIndex);
    PNew(maxIndex) = [];
    for k = 1:length(A) %aggiungo i K elementi "ma" con a \in A
       M{length(M)+1} = strcat(m,A{k});
       PNew = [PNew', p*P(k)]';
    end
end

T = M;
\end{verbatim}

        \color{lightgray} \begin{verbatim}Not enough input arguments.

Error in tunstall (line 6)
M = A;
\end{verbatim} \color{black}
    

\subsection*{Conversione in binario}

\begin{verbatim}
n = length(T);
l = ceil(log2(n));
    for i = 1:n
        T{i,2} = de2bi(i-1,l);
    end
\end{verbatim}
\begin{verbatim}
end
\end{verbatim}



\end{document}
    
